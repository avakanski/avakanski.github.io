
@Article{materials17153673,
AUTHOR = {Karimzadeh, Mohammad and Basvoju, Deekshith and Vakanski, Aleksandar and Charit, Indrajit and Xu, Fei and Zhang, Xinchang},
TITLE = {Machine Learning for Additive Manufacturing of Functionally Graded Materials},
JOURNAL = {Materials},
VOLUME = {17},
YEAR = {2024},
NUMBER = {15},
ARTICLE-NUMBER = {3673},
URL = {https://www.mdpi.com/1996-1944/17/15/3673},
PubMedID = {39124337},
ISSN = {1996-1944},
ABSTRACT = {Additive Manufacturing (AM) is a transformative manufacturing technology enabling direct fabrication of complex parts layer-by-layer from 3D modeling data. Among AM applications, the fabrication of Functionally Graded Materials (FGMs) has significant importance due to the potential to enhance component performance across several industries. FGMs are manufactured with a gradient composition transition between dissimilar materials, enabling the design of new materials with location-dependent mechanical and physical properties. This study presents a comprehensive review of published literature pertaining to the implementation of Machine Learning (ML) techniques in AM, with an emphasis on ML-based methods for optimizing FGMs fabrication processes. Through an extensive survey of the literature, this review article explores the role of ML in addressing the inherent challenges in FGMs fabrication and encompasses parameter optimization, defect detection, and real-time monitoring. The article also provides a discussion of future research directions and challenges in employing ML-based methods in the AM fabrication of FGMs.},
DOI = {10.3390/ma17153673}
}