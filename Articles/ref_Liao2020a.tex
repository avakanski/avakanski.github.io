@article{LIAO2020103687,
title = "A review of computational approaches for evaluation of rehabilitation exercises",
journal = "Computers in Biology and Medicine",
volume = "119",
pages = "103687",
year = "2020",
issn = "0010-4825",
doi = "https://doi.org/10.1016/j.compbiomed.2020.103687",
url = "http://www.sciencedirect.com/science/article/pii/S0010482520300780",
author = "Yalin Liao and Aleksandar Vakanski and Min Xian and David Paul and Russell Baker",
keywords = "Physical rehabilitation, Motion capture sensors, Rehabilitation datasets, Movement evaluation methods",
abstract = "Recent advances in data analytics and computer-aided diagnostics stimulate the vision of patient-centric precision healthcare, where treatment plans are customized based on the health records and needs of every patient. In physical rehabilitation, the progress in machine learning and the advent of affordable and reliable motion capture sensors have been conducive to the development of approaches for automated assessment of patient performance and progress toward functional recovery. The presented study reviews computational approaches for evaluating patient performance in rehabilitation programs using motion capture systems. Such approaches will play an important role in supplementing traditional rehabilitation assessment performed by trained clinicians, and in assisting patients participating in home-based rehabilitation. The reviewed computational methods for exercise evaluation are grouped into three main categories: discrete movement score, rule-based, and template-based approaches. The review places an emphasis on the application of machine learning methods for movement evaluation in rehabilitation. Related work in the literature on data representation, feature engineering, movement segmentation, and scoring functions is presented. The study also reviews existing sensors for capturing rehabilitation movements and provides an informative listing of pertinent benchmark datasets. The significance of this paper is in being the first to provide a comprehensive review of computational methods for evaluation of patient performance in rehabilitation programs."
}