@article{Vakanski2020,
title = "Attention enriched deep learning model for breast tumor segmentation in ultrasound images",
journal = "Ultrasound in Medicine and Biology",
volume = "46",
pages = "2819-2833",
year = "2020",
doi = "https://doi.org/10.1016/j.ultrasmedbio.2020.06.015",
url = "https://www.sciencedirect.com/science/article/abs/pii/S0301562920302878",
author = "Aleksandar Vakanski and Min Xian and Pheobe E. Freer",
keywords = "Physical rehabilitation, Motion capture sensors, Rehabilitation datasets, Movement evaluation methods",
abstract = "Incorporating human domain knowledge for breast tumor diagnosis is challenging because shape, boundary, curvature, intensity or other common medical priors vary significantly across patients and cannot be employed. This work proposes a new approach to integrating visual saliency into a deep learning model for breast tumor segmentation in ultrasound images. Visual saliency refers to image maps containing regions that are more likely to attract radiologists’ visual attention. The proposed approach introduces attention blocks into a U-Net architecture and learns feature representations that prioritize spatial regions with high saliency levels. The validation results indicate increased accuracy for tumor segmentation relative to models without salient attention layers. The approach achieved a Dice similarity coefficient (DSC) of 90.5% on a data set of 510 images. The salient attention model has the potential to enhance accuracy and robustness in processing medical images of other organs, by providing a means to incorporate task-specific knowledge into deep learning architectures."
}