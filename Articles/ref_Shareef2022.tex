@article{Shareef2022,
title = "ESTAN: Enhanced Small Tumor-Aware Network for Breast Ultrasound Image Segmentation",
journal = "Healthcare",
volume = "10",
pages = "1-14",
year = "2022",
doi = " https://doi.org/10.3390/healthcare10112262",
url = "https://www.mdpi.com/2227-9032/10/11/2262",
author = "Bryar Shareef and Aleksandar Vakanski and Pheobe E. Freer and Min Xian",
keywords = "Breast ultrasound; tumor segmentation; deep learning; small tumor-aware network",
abstract = "Breast tumor segmentation is a critical task in computer-aided diagnosis (CAD) systems for breast cancer detection because accurate tumor size, shape, and location are important for further tumor quantification and classification. However, segmenting small tumors in ultrasound images is challenging due to the speckle noise, varying tumor shapes and sizes among patients, and the existence of tumor-like image regions. Recently, deep learning-based approaches have achieved great success in biomedical image analysis, but current state-of-the-art approaches achieve poor performance for segmenting small breast tumors. In this paper, we propose a novel deep neural network architecture, namely the Enhanced Small Tumor-Aware Network (ESTAN), to accurately and robustly segment breast tumors. The Enhanced Small Tumor-Aware Network introduces two encoders to extract and fuse image context information at different scales, and utilizes row-column-wise kernels to adapt to the breast anatomy. We compare ESTAN and nine state-of-the-art approaches using seven quantitative metrics on three public breast ultrasound datasets, i.e., BUSIS, Dataset B, and BUSI. The results demonstrate that the proposed approach achieves the best overall performance and outperforms all other approaches on small tumor segmentation. Specifically, the Dice similarity coefficient (DSC) of ESTAN on the three datasets is 0.92, 0.82, and 0.78, respectively; and the DSC of ESTAN on the three datasets of small tumors is 0.89, 0.80, and 0.81, respectively."
}